\documentclass[tikz,border=10pt]{standalone}

\usepackage{pgfplots,amsmath}
\usetikzlibrary{angles,quotes}
\pgfplotsset{compat=newest}

\begin{document}
\pgfmathsetmacro\radius{2}

% Colors
\definecolor{tangentLineColor}{HTML}{BC5090}
\definecolor{remainingArcColor}{HTML}{003F5C}
\definecolor{involuteSplineColor}{HTML}{58508D}
\definecolor{accentColor}{HTML}{FF6361}
\colorlet{dashedLineColor}{black}

% Styles
\tikzstyle{information text}=[fill=red!10,inner sep=1ex]
\pgfplotsset{
   /pgf/number format/textnumber/.style={
     fixed,
     fixed zerofill,
     precision=3,
     1000 sep={,},
     },
  }

\foreach \rollAngle in {0.2,0.4,...,0.8}
{
	\begin{tikzpicture}
		[
			point/.style = {draw, circle, fill = black, inner sep = 1pt},
			dot/.style   = {draw, circle, fill = black, inner sep = .2pt},
			declare function = {
				involutex(\radius,\psi) = \radius * (cos(\psi) + \psi * sin(\psi));
				involutey(\radius,\psi) = \radius * (sin(\psi) - \psi * cos(\psi));
				arcx(\radius,\a,\psi) = \a + \radius * cos(\psi);
				arcy(\radius,\b,\psi) = \b + \radius * sin(\psi);
			},
		]
		\begin{axis}[
				name=plotAxis,
				trig format=rad,
				axis equal,
				axis lines=center,
				grid=both,
				xlabel=$x$, 
				ylabel=$y$,
				xmin=-5,xmax=5,
				ymin=-3,ymax=7,
			]
																					            
			\coordinate (O) at (0,0);
			\coordinate (Or) at (\radius,0);
			\coordinate (L1) at ({arcx(\radius,0,\rollAngle)},{arcy(\radius,0,\rollAngle)});
			\coordinate (L2) at ({involutex(\radius,\rollAngle)},{involutey(\radius,\rollAngle)});
						
			\addplot [domain=0:\rollAngle,samples=200,dashedLineColor,dashed,line cap=round]({arcx(\radius,0,x)},{arcy(\radius,0,x)});
			\addplot [domain=0.01:\rollAngle,samples=200,involuteSplineColor,thick,line cap=round]({involutex(\radius,x)},{involutey(\radius,x)});
			\addplot [domain=2*pi:\rollAngle,samples=200,remainingArcColor,thick,line cap=round]({arcx(\radius,0,x)},{arcy(\radius,0,x)});
																
			\draw[dashedLineColor,dashed] (O) -- (L1) node [pos=0.5,sloped,above] {$r$};
			\draw[tangentLineColor,thick] (L1) -- (L2);

			% \pic [draw,"$\psi$"] {angle = L1--O--Or};
												
		\end{axis}

		\node [xshift=.5cm,below right,align=center,text width=6cm,style=information text] at (plotAxis.north east)
		{
			\pgfmathsetmacro\rollAngleDeg{deg(\rollAngle)}
			\pgfmathsetmacro\arcLength{0.5 * \rollAngle * \radius^2}
			\pgfmathsetmacro\curvature{1 / (\radius * \rollAngle)}

			This is a demonstration how the {\color{accentColor} involute of a circle} work.
			So, {\color{accentColor} $r$} is radius of the circle, {\color{accentColor} $\psi$} --- roll angle,
			{\color{accentColor} $L$} --- arc length and {\color{accentColor} $k$} -- curvature of
			the involute.
			\begin{align*}
				{\color{accentColor} r} &= const & &= \radius \\
				{\color{accentColor} \psi} &= \pgfmathprintnumber[textnumber]{\rollAngle}\text{ rad}  & &\approx \pgfmathprintnumber[textnumber]{\rollAngleDeg}^\circ \\
				{\color{accentColor} L} &= \frac{1}{2} \psi r^2 & &\approx \pgfmathprintnumber[textnumber]{\arcLength} \\
				{\color{accentColor} \kappa} &= \frac{1}{\psi r} & &\approx \pgfmathprintnumber[textnumber]{\curvature}
			\end{align*}
		};

	\end{tikzpicture}
}
\end{document}